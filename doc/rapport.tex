\documentclass[a4paper,12pt]{article}
\usepackage{général}
%
\usepackage[french]{babel}
% \usepackage[margin=3cm,tmargin=5cm,bmargin=3.5cm]{geometry}
\usepackage{color}

\usepackage{float} % forcer impérativement le placement d’un flottant (figure ou table)

\newcommand\refsuscrite[1]{\textsuperscript{\ref{#1}}}

\title{
    \begin{figure}[!t]
	\begin{minipage}{.25\textwidth}
	    \includegraphics[width=\textwidth]{img/logo_lmu.png}
	\end{minipage}
	\hspace{.5\textwidth}
	\begin{minipage}{.25\textwidth}
	    \includegraphics[width=\textwidth]{img/logo_ic2.png}
	\end{minipage}
    \end{figure}
    \begin{center}
	\textbf{\textcolor{blue}{Le Mans Université}} \\
	Licence informatique 2\textsuperscript{e} année \\
	Module 174UP02 – Rapport de projet \\
	\textbf{Dig \& Rush}
    \end{center}
}
\author{
	\begin{tabular}{rl}
	    Matthieu & \textsc{Boulanger} \\
	    Ania & \textsc{Garoui} \\
	    Yohan & \textsc{Harison} \\
	    Jacques-Gérard & \textsc{Mpondo Toutou}
	\end{tabular}
}
\date{\today}


% plan Piau-Toffolon
% introduction
% analyse et conception
%	présentation du jeu −> principales fonctionnalités du jeu (haut niveau)
%       principales structures de données
% réalisation/développement
%       architecture du jeu (schéma fichiers), tableau principaux fichiers


\begin{document}

\maketitle
\begin{center}
    \href{https://github.com/idlusen/dig-and-rush/}{Lien vers le dépôt du projet}
\end{center}
\newpage

\tableofcontents
\newpage

% \abstract{Ceci est le texte de mon résumé...}

\section{Introduction}
% Rédaction : Yohan
% Longueur : 1 page

\textit{Cette introduction présentera le sujet qui sera traité et le travail avec une présentation du plan adopté} \\
Ce document présente un projet de jeu XXX réalisé dans le cadre de la formation de L2 informatique de l’université du Mans pendant la d-période de janvier à avril 2024.
Ce projet a été développé en langage C avec la librairie SDL.
Ce jeu fait ceci cela… \\
Nous présenterons dans une première partie notre jeu, des scénarios d’utilisation et les principales fonctionnalités puis dans une deuxième partie la gestion du projet ensuite dans une troisième partie les éléments principaux de conception (algorithmes, structures de données, …) ensuite nous présenterons l’architecture de notre application (structuration du code en fichiers) enfin nous montrerons les principaux résultats.
Enfin nous présenterons en conclusion les points forts et les limites de notre travail, les écarts entre la planification prévisionnelle et le déroulement réel de notre projet et les leçons tirés de cette expérience. En annexe nous présenterons un exemple de débogage et des tests (jeux d’essai et cas de test d’un exemple au moins - le fichier .c étant dans le git dans un répertoire dédié « test »)…



\section{Organisation}
% Rédaction : Yohan, Jacques
% Longueur : 2 pages
% Sujets : Gantt, répartition des tâches, salon de discussion, github (cartes ?)

\subsection{Sous-partie 1}

\subsection{Sous-partie 2}



\section{Conception}
% Rédaction : Yohan, Ania
% Longueur : 3 pages
% Sujets : analyse, cahier des charges etc.

\subsection{Sous-partie 1}

\subsection{Sous-partie 2}



\section{Développement}
% Rédaction : Matthieu, Ania, Jacques
% Longueur : 5 pages
% Sujets : librairies, architecture du code, menu, personnages, tour, blocs, documentation etc.

\subsection{Architecture}

\begin{table}[H]
    \centering
    \begin{tabular}{c p{.6\textwidth}}
	\toprule
	\texttt{main.c}			    & point d’entrée du programme	\\
	\midrule
	\texttt{ressources.c}		    & fonctions et structures de chargement des fichiers de ressources	\\
	\midrule
	\texttt{tour.c}			    & cœur du jeu avec notamment la boucle principale	\\
	\midrule
	\texttt{entite.c}                   & structure utilisée pour tout objet devant être affiché, déplacé et animé  \\
	\midrule
	\texttt{entite\_obstacle.c}          & spécialisation des entités qui agissent comme obstacles  \\
	\midrule
	\texttt{entite\_destructible.c}	    & spécialisation des entités qui peuvent être détruites \\
	\midrule
	\texttt{entite\_pnj.c}		    & spécialisation des entités représentant un personnage non joueur	\\
	\midrule
	\texttt{entite\_perso.c}		    & spécialisation des entités représentant le personnage du joueur	\\
	\midrule
	\texttt{texte.c}		    & API\refsuscrite{def_api} pour simplifier la gestion des textes   \\
	\midrule
	\texttt{nuit.c}			    & structure et fonctions implémentant une transition jour/nuit \\
	\bottomrule
    \end{tabular}
    \caption{Rôle des différents modules}
\end{table}

\subsection{Outils utilisés}



\section{Conclusion}
% Rédaction : Jacques, Matthieu
% Longueur : 1 page et demi



% Annexes
\newpage
\appendix

\section{Lexique}

\begin{table}[h]
    \centering
    \begin{tabular}{c p{.6\textwidth}}
	\toprule
	PNJ		    & Personnage Non Joueur \\
	\midrule
	API \label{def_api} & \textit{Application Programming Interface}, ensemble de classes et fonctions servant d’interface vers un service \\
	\bottomrule
    \end{tabular}
    \caption{Définitions des termes techniques employés dans le document}
\end{table}



\section{Tests unitaires}
\section{Exemple de débogage}

% Bibliographie
% \newpage
% \begin{thebibliography}{REF}
%     \bibitem{apa_scribbr}\url{https://www.scribbr.fr/category/normes-apa/}
%     \bibitem{apa_umontreal}\url{https://bib.umontreal.ca/citer/styles-bibliographiques/apa?tab=3281} 
%     \bibitem{wikibooks}\url{https://fr.wikibooks.org/wiki/LaTeX/Tableaux}
%     \bibitem{zestedesavoir}\url{https://zestedesavoir.com/tutoriels/826/introduction-a-latex/1322_completer-vos-documents/images-tableaux-et-texte-preformate/#2-tableaux}
% \end{thebibliography}

\end{document}
